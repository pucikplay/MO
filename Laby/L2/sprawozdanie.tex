\documentclass{article}
\usepackage[english]{babel}
\usepackage[letterpaper,top=2cm,bottom=2cm,left=3cm,right=3cm,marginparwidth=1.75cm]{geometry}
\usepackage{amsmath}
\usepackage{amssymb}
\usepackage{graphicx}
\usepackage[colorlinks=true, allcolors=blue]{hyperref}
\usepackage{polski}
\usepackage{enumitem}
\usepackage{float}
\usepackage[table]{xcolor}

\title{Metody optymalizacji L1}
\author{Gabriel Budziński\\254609}

\begin{document}
\maketitle

\section{Zadanie 1}
\subsection{Opis Modelu}
$w$ - wektor szerokości desek, $d$ - wektor zapotrzebowań, $p$ - macierz podziałów postaci $\mathbb{N}^{|w|\times k}$, gdzie $k \in \mathbb{N}$.
\subsubsection{Zmienne decyzyjne}
Zmienne decyzyjne mają postać wektora $\textit{x}$ spełniającego nierówność $\textit{x} \geq \textbf{0}$ o długości odpowiadającej liczbie możliwych cięć deski.
\subsubsection{Ograniczenia}
W modelu występuje tylko jeden typ ograniczeń:

\[(\forall i \in [|w|]) \left(\textit{x} \cdot p_{*i} \geq d_i\right)\]
gdzie $\cdot$ to iloczyn skalarny

\subsubsection{Funkcja celu}
W zadanym problemie staramy się minimalizować odpady z cięcia, co sprowadza się do minimalizacji zużycia standardowych desek, a w takim razie funkcja celu, którą minimalizujemy ma postać

\[\sum_{i = 1}^{k}{\textit{x}_i}\]

\subsection{Wyniki i interpretacja}

Optymalnym rozwiązaniem jest

\begin{table}[H]
	\centering
	\begin{tabular}{c|c|c|c}
		liczba sztuk & liczba desek szerokości 7 & liczba desek szerokości 5 & liczba desek szerokości 3\\\hline
		37 & 2 & 1 & 1\\\hline
		28 & 1 & 3 & 0\\\hline
		9 & 1 & 0 & 5\\\hline
	\end{tabular}
\end{table}

co daje odpowiednio 111,121 oraz 82 deski zadanych szerokości, a odpad wyniósł 18 cali.

\section{Zadanie 2}
\subsection{Opis Modelu}

\subsubsection{Zmienne decyzyjne}

\subsubsection{Ograniczenia}

\subsubsection{Funkcja celu}

\subsection{Wyniki i interpretacja}

\section{Zadanie 3}
\subsection{Opis Modelu}

\subsubsection{Zmienne decyzyjne}

\subsubsection{Ograniczenia}

\subsubsection{Funkcja celu}

\subsection{Wyniki i interpretacja}

\section{Zadanie 4}
\subsection{Opis Modelu}

\subsubsection{Zmienne decyzyjne}

\subsubsection{Ograniczenia}

\subsubsection{Funkcja celu}

\subsection{Wyniki i interpretacja}
\end{document}