\documentclass{article}
\usepackage[english]{babel}
\usepackage[letterpaper,top=2cm,bottom=2cm,left=3cm,right=3cm,marginparwidth=1.75cm]{geometry}
\usepackage{amsmath}
\usepackage{amssymb}
\usepackage{graphicx}
\usepackage[colorlinks=true, allcolors=blue]{hyperref}
\usepackage{polski}
\usepackage{enumitem}
\usepackage{float}
\usepackage[table]{xcolor}
\usepackage{tikz}
\usepackage{pgfgantt}

\newcommand\Dganttbar[4]{%
  \ganttbar{#1}{#3}{#4}\ganttbar[inline,bar label font=\footnotesize]{#2}{#3}{#4}
}

\title{Metody optymalizacji L1}
\author{Gabriel Budziński\\254609}

\begin{document}
\maketitle

\section{Zadanie 1}
\subsection{Opis Modelu}
$w$ - wektor szerokości desek, $d$ - wektor zapotrzebowań, $p$ - macierz podziałów postaci $\mathbb{N}^{|w|\times k}$, gdzie $k \in \mathbb{N}$.
\subsubsection{Zmienne decyzyjne}
Zmienne decyzyjne mają postać wektora $\textit{x}$ spełniającego nierówność $\textit{x} \geq \textbf{0}$ o długości odpowiadającej liczbie możliwych cięć deski.
\subsubsection{Ograniczenia}
W modelu występuje tylko jeden typ ograniczeń:

\[(\forall i \in [|w|]) \left(\textit{x} \cdot p_{*i} \geq d_i\right)\]
gdzie $\cdot$ to iloczyn skalarny

\subsubsection{Funkcja celu}
W zadanym problemie staramy się minimalizować odpady z cięcia, co sprowadza się do minimalizacji zużycia standardowych desek, a w takim razie funkcja celu, którą minimalizujemy ma postać

\[\sum_{i = 1}^{k}{\textit{x}_i}\]

\subsection{Wyniki i interpretacja}

Optymalnym rozwiązaniem jest

\begin{table}[H]
	\centering
	\begin{tabular}{c|c|c|c}
		liczba sztuk & liczba desek szerokości 7 & liczba desek szerokości 5 & liczba desek szerokości 3\\\hline
		37 & 2 & 1 & 1\\\hline
		28 & 1 & 3 & 0\\\hline
		9 & 1 & 0 & 5\\\hline
	\end{tabular}
\end{table}

co daje odpowiednio 111,121 oraz 82 deski zadanych szerokości, a odpad wyniósł 18 cali.

\section{Zadanie 2}
\subsection{Opis Modelu}
$n$ - liczba zadań, $p$ - wektor czasów wykonania zadań $\in \mathbb{R}^n$, $r$ - wektor czasów gotowości zadania $\in \mathbb{R}^n$, $w$ - wektor wag zadań $\in \mathbb{R}^n$, $M$ - duża liczba (np. $M > \sum_{i=1}^{n}{p_i}$)
\subsubsection{Zmienne decyzyjne}
\begin{itemize}
	\item $c$ - wektor czasów zakończenia zadań $\in \mathbb{R}^n$
\end{itemize}

\subsubsection{Ograniczenia}
\begin{itemize}
	\item spełnienie warunków gotowości
	\[(\forall i in [n])\left(c_i - p_i \geq r_i\right)\]
	\item wymuszenie rozłączności zadań na tej samej maszynie
	\[(\forall i,j \in [n])\left(i < j \implies c_i \leq c_j - p_j + M(1-y_{ij})\right)\]
	\[(\forall i,j \in [n])\left(i < j \implies c_j \leq c_i - p_i + My_{ij}\right)\]
\end{itemize}

\subsubsection{Funkcja celu}
Funkcja celu, którą minimalizujemy to
\[\sum_{i=1}^{n}{c_i w_i}\]

\subsection{Wyniki i interpretacja}
Dla przykładowych danych $n = 3, p = (2,3,2), w = (1,1,5), r = (2,3,5)$ program zwrócił rozwiązanie postaci

\begin{figure}[H]
	\begin{ganttchart}[
	bar/.append style={fill=blue!50},
	expand chart=\textwidth]{0}{9}
	\gantttitlelist{0,...,9}{1} \\
	\Dganttbar{}{1}{2}{3}
	\Dganttbar{}{2}{7}{9}
	\Dganttbar{}{3}{5}{6}
	\end{ganttchart}
\end{figure}

O wartości funkcji celu 49.

\section{Zadanie 3}
\subsection{Opis Modelu}
$n$ - liczba zadań, $m$ - liczba maszyn, $p$ - wektor czasów wykonania zadań $\in \mathbb{R}^n$, $r$ - macierz relacji poprzedzania $\in \{0,1\}^{n\times n}$, $M$ - duża liczba (np. $M > \sum_{i=1}^{n}{p_i}$)

\subsubsection{Zmienne decyzyjne}
\begin{itemize}
	\item $s$ - wektor czasów rozpoczęcia zadań $\in \mathbb{R}^n$
	\item $c$ - wektor numerów maszyn, na których wykonano dane zdanie $\in [m]^n$
	\item $y$ - binarna macierz pomocnicza do wymuszania rozłączności zadań na danej maszynie $\in \{0,1\}^{n\times n}$
	\item $x$ - binarna macierz pomocnicza do sprawdzania, czy zadania są wykonywane na tej samej maszynie $\in \{0,1\}^{n\times n\times 3}$
	\item $C$ - ograniczenie górne czasów zakończenia zadań $\in \mathbb{R}$
\end{itemize}

\subsubsection{Ograniczenia}
\begin{itemize}
	\item ustawienie wartości $x$
	\[(\forall i,j \in [n])\left(i < j \implies Lx_{ij1} + bx_{ij2} + (b+\delta)x_{ij3} \leq c[i] - c[j] \leq (b-\delta)x_{ij1} + bx_{ij2}  + Ux_{ij3}\right)\] gdzie $L=-m, U=m, \delta=0.1, b=0$
	\[(\forall i,j \in [n])\left(\sum_{k\in \{1,2,3\}}{x_{i,j,k}} = 1\right)\]
	\item wymuszenie rozłączności zadań na tej samej maszynie
	\[(\forall i,j \in [n])\left(i < j \implies s_i + M(y_{ij} + (1-x_{ij2})) \geq s_j + p_j\right)\]
	\[(\forall i,j \in [n])\left(i < j \implies s_j + M((1-y_{ij}) + (1-x_{ij2})) \geq s_i + p_i\right)\]
	\item wymuszenie zadanego poprzedzania
	\[(\forall i,j \in [n])\left(i < j \and r_{ij} = 1 \implies s_i + p_i \leq s_j\right)\]
	\item ustawienie $C$ jako ograniczenia górnego na czas zakończenia
	\[(\forall i \in [n])\left(C \geq s_i + p_i\right)\]
\end{itemize}

\subsubsection{Funkcja celu}
Funkcja celu przyjmuje wartość $C$, które staramy się zminimalizować.

\subsection{Wyniki i interpretacja}

Po optymalizacji model odpowiada rozwiązaniu,

\begin{figure}[H]
	\begin{ganttchart}[
	bar/.append style={fill=blue!50},
	expand chart=\textwidth]{0}{8}
	\gantttitlelist{0,...,8}{1} \\
	\Dganttbar{M1}{3}{0}{0}
	\Dganttbar{}{5}{2}{2}
	\Dganttbar{}{8}{3}{8} \\
	\Dganttbar{M2}{2}{0}{1}
	\Dganttbar{}{6}{4}{4} \\
	\Dganttbar{M3}{1}{0}{0}
	\Dganttbar{}{4}{2}{3}
	\Dganttbar{}{7}{4}{6}
	\Dganttbar{}{9}{7}{8}
	\end{ganttchart}
\end{figure}

które spełnia zadane założenia i ma wartość funkcji celu $C_{max} = 9$.

\section{Zadanie 4}
\subsection{Opis Modelu}

\subsubsection{Zmienne decyzyjne}

\subsubsection{Ograniczenia}

\subsubsection{Funkcja celu}

\subsection{Wyniki i interpretacja}
\end{document}