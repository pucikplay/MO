\documentclass{article}
\usepackage[english]{babel}
\usepackage[letterpaper,top=2cm,bottom=2cm,left=3cm,right=3cm,marginparwidth=1.75cm]{geometry}
\usepackage{amsmath}
\usepackage{amssymb}
\usepackage{graphicx}
\usepackage[colorlinks=true, allcolors=blue]{hyperref}
\usepackage{polski}
\usepackage{enumitem}
\usepackage{float}

\title{Metody optymalizacji L1}
\author{Gabriel Budziński\\254609}

\begin{document}
\maketitle

\section{Zadanie 1}
\subsection{Opis Modelu}
\subsubsection{Zmienne decyzyjne}
Zmienne decyzyjne mają postać wektora $\textbf{\textit{x}}$ spełniającego nierówność $\textbf{\textit{x}} \geq \textbf{0}$.
\subsubsection{Ograniczenia}
Zadany jest zestaw równań liniowych postaci
\[A\textbf{\textit{x}}=\textbf{\textit{b}}\]

gdzie

\[a_{ij} = \frac{1}{i+j-1}, i,j \in [n]\]
\[c_i = b_i = \sum_{j=1}^n{\frac{1}{i+j-1}}, i \in [n]\]

\subsubsection{Funkcja celu}
Funkcja celu ma postać min ${\textit{{\textbf{c}}}}^T \textbf{\textit{x}}$, gdzie $\textbf{\textit{c}}$ to wektor współczynników kosztu.
\subsection{Wyniki i interpretacja}
Prawidłowym rozwiązaniem zadania jest $\textit{\textbf{x}}=\textbf{1}$, ale przez macierz A zadanie jest źle uwarunkowane.
Rozwiązano model dla wartości $n$ ze zbioru $\{1,2,\dots,10\}$:

\begin{table}[H]
\centering
\begin{tabular}{|l|l|}\hline
n & $||x-\tilde{x}||_2/||x||_2$\\\hline
2 & $1.10933564796705\cdot 10^{-30}$\\\hline
3 & $1.34804824480736\cdot 10^{-29}$\\\hline
4 & $1.06939716107966\cdot 10^{-25}$\\\hline
5 & $1.12318763722721\cdot 10^{-23}$\\\hline
6 & $4.66947802820021\cdot 10^{-21}$\\\hline
7 & $2.81798474577652\cdot 10^{-16}$\\\hline
8 & 0.26425662687595\\\hline
9 & 0.466367895688764\\\hline
10 & 0.980867548324338\\\hline
\end{tabular}
\end{table}

Jak widzimy, błąd względny jest niewielki dla $x=2$, ale rośnie coraz szybciej i dla $x=10$ jest już ponad 1\%. Z dokładnością do co najmniej dwóch cyfr można obliczyć dla $n \leq 7$.

\section{Zadanie 2}
\subsection{Opis Modelu}
\subsubsection{Zmienne decyzyjne}
W modelu mamy trójwymiarową macierz zmiennych decyzyjnych $x \in \mathbb{R}^{n\times n\times 2}$, gdzie $x_{ijk}$ oznacza liczbę dźwigów typu $k$ przetransportowanych z miasta $i$ do miasta $j$.
\subsubsection{Ograniczenia}
Zgodnie z treścią zadania zaprogramowano trzy ograniczenia:
\begin{itemize}
	\item Z miasta nie może wyjechać więcej dźwigów typu $k$ niż jest ich w nadmiarze (dwuwymiarowa macierz $s \in \mathbb{N}^{n\times 2}$, zatem
	\[(\forall i \in {0,\dots,n-1}, k \in \{I,II\}) \left(\sum_{j=0}^{n-1}{x_{ijk} \leq s_{ik}}\right)\]
	\item Każde miasto powinno dostać co najmniej tyle dźwigów typu II ile jest deficytu (dwuwymiarowa macierz $d \in \mathbb{N}^{n\times 2}$)
	\[(\forall i \in {0,\dots,n-1}) \left(\sum_{j=0}^{n-1}{x_{jiII} \geq d_{iII}}\right)\]
	\item Suma wszystkich dźwigów przetransportowanych do miasta powinna być równa co najmniej sumie deficytów wszystkich typów
	\[(\forall i \in {0,\dots,n-1}) \left(\sum_{j=0,k=I}^{n-1,II}{x_{jik}} \geq \sum_{k=I}^{II}{d_{ik}}\right)\]
\end{itemize}
\subsubsection{Funkcja celu}
Funkcją celu jest koszt transportu dźwigów pomiędzy miastami, który należy zminimalizować. Do rozwiązania modelu wprowadzono odległości między podanymi miastami pobrane z Google Maps.
\subsection{Wyniki i interpretacja}

W poniższych tabelach przedstawiono obliczone wartości zmiennych decyzyjnych. W kolumnach wypisano miasta do których transportowane były dźwigi, a w wierszach te, z których te dźwigi pochodziły.

\begin{table}[H]
\centering
\begin{tabular}{l|c c c c c c c }
 & Opole & Brzeg & Nysa & Prudnik & Strzelce & Koźle & Racibórz\\\hline
Opole & - & 4 & - & - & - & 3 & -\\\hline
Brzeg & - & - & - & - & - & - & -\\\hline
Nysa & - & 5 & - & 1 & - & - & -\\\hline
Prudnik & - & - & - & - & - & - & -\\\hline
Strzelce & - & - & - & - & - & 5 & -\\\hline
Koźle & - & - & - & - & - & - & -\\\hline
Racibórz & - & - & - & - & - & - & -\\\hline
\end{tabular}
\caption{Dźwigi typu I}
\end{table}

\begin{table}[H]
\centering
\begin{tabular}{l|c c c c c c c}
 & Opole & Brzeg & Nysa & Prudnik & Strzelce & Koźle & Racibórz\\\hline
Opole & - & - & - & - & - & - & -\\\hline
Brzeg & - & 1 & - & - & - & - & -\\\hline
Nysa & 2 & - & - & - & - & - & -\\\hline
Prudnik & - & - & - & 3 & 4 & 2 & 1\\\hline
Strzelce & - & - & - & - & - & - & -\\\hline
Koźle & - & - & - & - & - & - & -\\\hline
Racibórz & - & - & - & - & - & - & -\\\hline
\end{tabular}
\caption{Dźwigi typu II}
\end{table}

Ograniczenie całkowitoliczbowości nie jest konieczne, ponieważ model zwraca to samo rozwiązanie z nim oraz bez niego.

\section{Zadanie 3}
\subsection{Opis Modelu}
\subsubsection{Zmienne decyzyjne}
\subsubsection{Ograniczenia}
\subsubsection{Funkcja celu}
\subsection{Wyniki i interpretacja}

\section{Zadanie 4}
\subsection{Opis Modelu}
\subsubsection{Zmienne decyzyjne}
\subsubsection{Ograniczenia}
\subsubsection{Funkcja celu}
\subsection{Wyniki i interpretacja}

\end{document}