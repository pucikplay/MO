\documentclass{article}
\usepackage[english]{babel}
\usepackage[letterpaper,top=2cm,bottom=2cm,left=3cm,right=3cm,marginparwidth=1.75cm]{geometry}
\usepackage{amsmath}
\usepackage{amssymb}
\usepackage{graphicx}
\usepackage[colorlinks=true, allcolors=blue]{hyperref}
\usepackage{polski}
\usepackage{enumitem}
\usepackage{float}
\usepackage[table]{xcolor}
\usepackage{tikz}

\title{Problem k-minimalnego drzewa rozpinającego}
\author{Gabriel Budziński\\254609}

\begin{document}
\maketitle

\section{Wprowadzenie}

Weźmy graf nieskierowany $G=(V,E)$ o $n$ wierzchołkach $w \in V$, nieujemnych kosztach $c_e$ krawędzi $e \in E$ oraz liczbę $k \in \mathbb{N}$. Problem k-minimalnego drzewa rozpinającego ($\textit{ang.}$ kMST - $k$-minimal spanning tree, MSkT - minimal spanning $k$-tree) polega na poszukiwaniu drzewa w $G$ o minimalnym koszcie, w które wchodzi co najmniej $k$ wierzchołków $G$. Problem ten jest NP-trudny nawet dla $V$ należących do płaszczyzny Euklidejskiej. Problem ten jest silnie związany z podobnym problemem występującym we wcześniejszych latach w literaturze - $k$-CARD TREE, który sprowadza się do znalezienia w grafie $G$ poddrzewa o $k$ krawędziach, który to problem również jest NP-trudny.

\end{document}