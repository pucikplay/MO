\documentclass{article}
\usepackage[english]{babel}
\usepackage[letterpaper,top=2cm,bottom=2cm,left=3cm,right=3cm,marginparwidth=1.75cm]{geometry}
\usepackage{amsmath}
\usepackage{amssymb}
\usepackage{graphicx}
\usepackage[colorlinks=true, allcolors=blue]{hyperref}
\usepackage{polski}
\usepackage{enumitem}
\usepackage{float}
\usepackage[table]{xcolor}
\usepackage{tikz}
\usepackage{floatrow}

\title{Problem k-minimalnego drzewa rozpinającego}
\author{Gabriel Budziński\\254609}

\begin{document}
\maketitle

\section{Wprowadzenie}

Weźmy graf nieskierowany $G=(V,E)$ o $n$ wierzchołkach $w \in V$, nieujemnych kosztach $c_e$ krawędzi $e \in E$ oraz liczbę $k \in \mathbb{N}$. Problem k-minimalnego drzewa rozpinającego ($\textit{ang.}$ kMST - $k$-minimal spanning tree, MSkT - minimal spanning $k$-tree) polega na poszukiwaniu drzewa w $G$ o minimalnym koszcie, w które wchodzi co najmniej $k$ wierzchołków $G$. Problem ten jest NP-trudny nawet dla $V$ należących do płaszczyzny Euklidejskiej. Problem ten jest silnie związany z innym, występującym we wcześniejszych latach w literaturze \cite{k_card_trees} - minimum weight $k$-cardinality tree, którego rozwiązaniem jest znalezienie w grafie $G$ poddrzewa o $k$ krawędziach.

\section{$k$-cardinality tree}

\subsection{Opis problemu}
Weźmy graf $G = (V,E)$ ze zbiorem wierzchołków $V$ i krawędzi $E$. Moce zbiorów $V$ i $E$ to odpowiednio $n = |V|$ oraz $m = |E|$. Dla każdej krawędzi $e \in E$ dana jest waga $w(e) \in \mathbb{R}$, a waga zbioru $E' \subseteq E$ jest definiowana jako $\sum_{e \in E'}{w(e)}$.

Drzewem w $G$ jest podgraf $T = (V(T), E(T))$ taki, że $T$ nie zawiera cykli i jest spójny. Będziemy używać notacji $w(T)$ opisując $w(E(T))$. Moc $|T|$ zbioru $T$ jest mocą $E(T)$. Dla zadanego $k$, gdzie $1 \leq k \leq n-1$ $k$-cardinality tree jest drzewem $T$ o mocy $|T| = k$. Jeśli $k = n-1$ to $T$ jest drzewem rozpinającym $G$. Zadane jest znalezienie takiego $T$, że $w(T) = \min_{T' \subseteq G}{w(T')}$. Dla $k = n-1$ takim $T$ jest minimalne drzewo rozpinające, które można znaleźć w czasie wielomianowym algorytmem zachłannym (Kruskal \cite{kruskal}, Prim \cite{prim}). Dla ustalonego $k$ problem jest różnież rozwiązywalny przez wyliczenie możliwych drzew.

\subsection{Zastosowania w praktyce}

Powyższy problem pojawia się w najmie pól naftowych \cite{oil_fields}. Rząd ma następującą regułę "50\%" obejmującą morskie pola naftowe: jeśli firma najęła pole naftowe ma ona ustaloną liczbę lat, dajmy na to 5, aby eksploatować to pole. Po upływie tego czasu firma ma obowiązek zwrócić co najmniej 50\% najętego pola. Ponadto, oddawana część pola musi być spójna. Oczywistym celem z punktu widzenia firmy jest zwrot częsci o najmniejszej wartości (i zachowanie części o wartości największej). W pracy \cite{oil_fields} pola naftowe mają postać prostokąta podzielonego na mniejsze kwadraty. Firma, która najmuje pole ma 5 lat na zebranie informacji o wartości $w_i$ każdego z podkwadratów. Część pola, którą firma odda odpowiada podzbiorowi co najmniej 50\% podkwadratów, który jest spójny i ma najmniejszą całkowitą wartość wszystkich $w_i$. Aby zamodelować spójność weźmy graf dualny do oczekiwanego, który jest grafem kratowym.

\begin{figure}
    \ffigbox{%
      \rule{3cm}{3cm}%
    }{%
      \caption{A figure}%
    }
    \capbtabbox{%
      \begin{tabular}{cc} \hline
      Author & Title \\ \hline
      Knuth & The \TeX book \\
      Lamport & \LaTeX \\ \hline
      \end{tabular}
    }{%
      \caption{A table}%
    }
\end{figure}

\begin{table}
    \centering
    \begin{tabular}{|c|c|c|c|c|}\hline
        1 & 2 & 3 & 4 & 5\\\hline
        6 & 7 & 8 & 9 & 10\\\hline
        11 & 12 & 13 & 14 & 15\\\hline
    \end{tabular}
\end{table}

\begin{tikzpicture}
    % Set the size of each node
    \newcommand{\nodesize}{0.5cm}
  
    % Draw the grid graph
    \foreach \x in {0,1,2,3,4}
      \foreach \y in {0,1,2}
        \node [fill,minimum size=\nodesize] at (\x,\y) {};
  
    % Connect the nodes horizontally
    \foreach \x in {0,1,2,3,4}
      \foreach \y [count=\nexty from 1] in {0,1}
        \draw (\x,\y) -- (\x,\nexty);
  
    % Connect the nodes vertically
    \foreach \x [count=\nextx from 1] in {0,1,2,3}
      \foreach \y in {0,1,2}
        \draw (\x,\y) -- (\nextx,\y);      
\end{tikzpicture}

\section{$k$-spanning tree}

\newpage

\bibliography{bibliography}
\bibliographystyle{ieeetr}

\end{document}