\documentclass{article}
\usepackage[english]{babel}
\usepackage[letterpaper,top=2cm,bottom=2cm,left=3cm,right=3cm,marginparwidth=1.75cm]{geometry}
\usepackage{amsmath}
\usepackage{amssymb}
\usepackage{graphicx}
\usepackage[colorlinks=true, allcolors=blue]{hyperref}
\usepackage{polski}
\usepackage{enumitem}
\usepackage{float}

\title{Metody optymalizacji L3}
\author{Gabriel Budziński\\254609}

\begin{document}
\maketitle

\section{Zadanie 1}

\subsection*{Treść}

W zadaniu należało zaimplementować w języku \texttt{julia} z użyciem pakietu \texttt{JuMP} algorytm 2-aproksymacyjny oparty na programowaniu liniowym dla problemu szeregowania zadań na niezależnych maszynach z kryterium minimalizacji długości uszeregowania.

\subsection*{Algorytm}



\end{document}